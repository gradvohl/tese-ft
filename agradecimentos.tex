%% Preencha aqui a dedicatória, os agradecimentos e a epígrafe.

% Se houver uma dedicatória, informe no comando a seguir. 
% A dedicatória deve ter poucas linhas.
\dedicatoria{Adicione aqui um texto informando a quem você dedica este trabalho. A dedicatória deve ocupar poucas linhas.}

 %% Essa seção é obrigatória. Deve-se, pelo menos, agradecer à CAPES.
\prefacesection{Agradecimentos}
Coloque nesse ponto os agradecimentos àqueles que o ajudaram no seu trabalho. Os agradecimentos devem ocupar uma única página. 

\textbf{IMPORTANTE:} \textbf{\underline{Se for uma dissertação de mestrado ou tese de doutorado}, não esqueça de adicionar a frase a seguir, mesmo que você não tenha recebido bolsa CAPES}. 

O presente trabalho foi realizado com apoio da Coordenação de Aperfeiçoamento de Pessoal de Nível Superior -- Brasil (CAPES) -- Código de Financiamento 001.



% Se houver epígrafe, use o comando \epigrafe a seguir.
% Caso não haja epígrafe, remova o comando (até o final deste arquivo).
\epigrafe{%
Epígrafe é uma citação curta (frase, verso ou pensamento), colocada no início de uma obra, capítulo ou trabalho acadêmico, que serve como introdução, reflexão ou tema para o texto que se segue. 

  \begin{flushright}
    Autor da citação
  \end{flushright}
} % Fim epigrafe