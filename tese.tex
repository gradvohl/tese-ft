%%%%%%%%%%%%%%%%%%%%%%%%%%%%%%%%%%%%%%%%%%%%%%%%%%%%%%%%%%%%%%%%%%
% The following comments were written in Portuguese, because this 
% template applies only for School of Technology at University 
% of Campinas, Brazil.
%
% Este é um modelo Latex para monografias de Trabalhos de Conclusão 
% de Curso (TCC) na graduação, monografias de Mestrado e Teses de 
% doutorado da Faculdade de Tecnologia (FT) da Universidade 
% Estadual de Campinas (UNICAMP).
%
% Esse modelo e seu respectivo arquivo de classe de documento 
% foram adaptados do modelo de teses e dissertações do 
% Instituto de Computação da UNICAMP e estão de acordo com a 
% Instrução Normativa CPG 002/2021.
%
% Autor: André Leon Sampaio Gradvohl, Dr.
% Email:     gradvohl@unicamp.br
% Lattes CV: http://lattes.cnpq.br/9343261628675642
% ORCID:     0000-0002-6520-9740
% 
% Última versão: 7/junho/2025.
%
% Adições/Alterações nesta última versão:
% - Ajustes na folha de aprovação.
%   - Remoção da área de concentração quando for curso de 
%     graduação na folha de aprovação.
%   - Adaptação do texto na folha de aprovação quando se 
%     referir a graduação, mestrado ou a doutorado.
%%
%%%%%%%%%%%%%%%%%%%%%%%%%%%%%%%%%%%%%%%%%%%%%%%%%%%%%%%%%%%%%%%%%%
%
% Escolha: Portugues ou Ingles ou Espanhol.
% Para a versão final do texto, acrescente a palavra "Final".
\documentclass[Portugues,Final]{tese-FT}
%\documentclass[Ingles,Final]{tese-FT}
%\documentclass[Espanhol,Final]{tese-FT}
%
% Para uma compilação mais rápida, utilize a opção "Draft", em
% letras maiúsculas como a seguir. 
%\documentclass[Portugues,Draft]{tese-FT}
%
% Caso necessário, adicione também a opção "noFig". Essa opção 
% deixa de mostrar as figuras e, portanto, acelera mais ainda
% a compilação.
%\documentclass[Portugues,Draft,noFig]{tese-FT}
%
% Caso queira gerar uma versão para avaliação pelo Turnitin,
% (software utilizado pela Unicamp para emissão do relatório
% de originalidade), adicione a opção Turnitin conforme o 
% exemplo da linha a seguir:
%\documentclass[Portugues,Final,Turnitin]{tese-FT}

%Adicione seu arquivo com as referências bibliográficas
\addbibresource{bibliografia.bib}

%O pacote a seguir gera um "dummy text". Elimine a linha quando
% for editar seu texto.
\usepackage{lipsum}

\begin{document}

% Escolha entre autor ou autora:
\autor{Nome do autor}
%\autora{Nome da Autora}

% Sempre deve haver um título em português:
\titulo{Título da Dissertação ou Tese em Português}

% Se a língua for o inglês ou o espanhol defina:
%\title{The Dissertation or Thesis Title in English or Spanish for FT}

% Escolha entre orientador ou orientadora e inclua os títulos:
\orientador{Prof. Dr. Nome do Orientador}
%\orientadora{Profa. Dra. Nome da Orientadora}

% Escolha entre coorientador ou coorientadora, se houver, 
% e inclua os títulos:
%\coorientador{Prof. Dr. Eng. Lic. Nome do Co-Orientador}
%\coorientadora{Prof. Dra. Eng. Lic. Nome da Co-Orientadora}

% Escolha entre uma das seis opções a seguir (comente as demais):
%\bsi         % para Trabalho de Conclusão de Curso em BSI
%\tads        % para Trabalho de Conclusão de Curso em TADS
%\qualificacaoMestrado  % Para textos de qualificação de mestrado.
%\qualificacaoDoutorado % Para textos de qualificação de doutorado.
\mestrado   % para Dissertação de Mestrado em Tecnologia
%\doutorado  % para Tese de Doutorado em Tecnologia

%Defina a área de concentração. Se for TCC, deixe comentado.
\areaConcentracao{Sistemas de Informação e Comunicação}
%\areaConcentracao{Ambiente}
%\areaConcentracao{Ciência dos Materiais}

% Se houve cotutela, defina:
%\cotutela{Universidade Nova de Plutão}

% Defina a data da defesa no formato {Dia}{Mês}{Ano}
% Use apenas números! O template transformará em palavras,
% se necessário.
\datadadefesa{20}{5}{2025}

% Para a versão final defina:
% Repita o nome do Orientador(a) no primeiro avaliador
\avaliadorA{Prof. Dr. Nome do Orientador}{FT/UNICAMP}
\avaliadorB{Profa. Dra. Segunda Avaliadora}{Instituição da segunda avaliadora}
\avaliadorC{Dr. Terceiro Avaliador}{Instituição do terceiro avaliador}
% \avaliadorD{Prof. Dr. Quarto Avaliador}{Instituição do quarto avaliador}
% \avaliadorE{Prof. Dr. Quinto Avaliador}{Instituição do quinto avaliador}
% \avaliadorF{Prof. Dr. Sexto Avaliador}{Instituição do sexto avaliador}
% \avaliadorG{Prof. Dr. Sétimo Avaliador}{Instituição do sétimo avaliador}
% \avaliadorH{Prof. Dr. Oitavo Avaliador}{Instituição do oitavo avaliador}

% Para incluir a ficha catalográfica em PDF na versão final, 
% copie o arquivo PDF para o projeto no Overleaf, descomente 
% e informe o nome do arquivo no comando a seguir.
%\fichacatalografica{SeuArquivo.pdf}
% OBSERVAÇÂO: O comando para a ficha catalográfica só é válido 
% versoes finais de TCC, teses e dissertações (não é válido para qualificação 
% ou TCCs que não serão disponibilizados na biblioteca).
%
% Para deixar uma página em branco no lugar da ficha 
% catalográfica, descomente uma das três linhas a seguir:
%\fichacatalografica{branco.pdf} % Português
%\fichacatalografica{white.pdf}  % Inglês
%\fichacatalografica{blanco.pdf} % Espanhol

% Este comando deve ficar aqui:
\paginasiniciais

% Se houver uma dedicatória, informe no comando a seguir. 
% A dedicatória deve ter poucas linhas.
 %\dedicatoria{A dedicatória deve ocupar poucas linhas.}
 
% Se houver epígrafe, use o comando \epigrafe a seguir
%   - O primeiro parâmetro é o autor dos dizeres.
%   - O segundo parâmetro são os dizeres.
% \epigrafe{Hippocrates}{
% {\it
% Vita brevis,\\
% ars longa,\\
% occasio praeceps,\\
% experimentum periculosum,\\
% iudicium difficile.}
% }

% O comando condicional \ifturnitin a seguir é importante para 
% preparar o texto para encaminhamento ao Turnitin. 
% NÃO REMOVA!!
% O \fi correspondente está após o comando \tableofcontents
\ifturnitin
   \relax 
\else
% Adicione no arquivo "agradecimentos.tex" os seus agradecimentos
% Caso prefira omitir os agradecimentos, comente a linha a seguir.
\input{agradecimentos}

% Sempre deve haver um resumo em português:
\begin{resumo}
O resumo deve ter no máximo 500 palavras e deve ocupar uma única página.
\end{resumo}

% Sempre deve haver um abstract:
\begin{abstract}
The abstract must have at most 500 words and must fit in a single page.
\end{abstract}

% Se houver um resumo em espanhol, descomente as linhas a seguir:
%\begin{resumen}
% A mesma regra aplica-se.
%\end{resumen}

% A lista de figuras:
\listoffigures

% A lista de tabelas:
\listoftables

% A lista de símbolos é opcional. Não confunda a lista de símbolos 
% matemáticos com a lista de abreviaturas (que vem depois).
\input{listaSimbolos}

% A lista de abreviações e siglas vem a seguir.
% Dê uma olhada no pacote nomencl para ver os comandos para 
% adicionar abreviações e siglas no texto.
% Foram adicionados os comandos \Sigla{Sigla por extenso}{abrev} e 
% \SiglaHifen{Sigla por extenso}{abrev} para adicionar as siglas 
% diretamente no texto e criar a lisa de abreviaturas 
% automaticamente
\printnomenclature[3cm]

% O sumário vem aqui:
\tableofcontents

%% O comando \fi a seguir é obrigatório para o controle 
%% da opção "turnitin". Não o remova!!
\fi

% E a linha a seguir deve ficar bem aqui. Não mude.
\fimdaspaginasiniciais

% O corpo da dissertação ou tese começa aqui:
%
% O comando a seguir inclui o arquivo introducao.tex
% que contém o capítulo de Introdução. 
% Detalhe: não precisa incluir a extensão .tex
% Aqui começa o capítulo de Introdução.
% Use o comando \label para definir um rótulo, 
% caso seja necessário referenciar esse capítulo
% posteriormente.
\chapter{Introdução}\label{chp:Introducao}

Esse é um texto de exemplo para o modelo de teses e dissertações da Faculdade de Tecnologia (FT) da UNICAMP. Esse modelo se aplica às teses de doutorado, dissertações de mestrado e Trabalhos de Conclusão de Curso, para a graduação (dos cursos de informática da FT).

A seguir, estão alguns exemplos para os elementos mais comuns em textos acadêmicos, \ie tabelas, equações e figuras. 

\section{Exemplos de Tabelas e Equações}\label{sec:exemplostabelas}
Aqui um exemplo de como referenciar a \Tabela{tab:tabela_1}. Note que é preciso definir um rótulo (\textit{label}) dentro do comando de definição da tabela.

% Veja a seguir um exemplo de Tabela.
% Você pode usar o site http://www.tablesgenerator.com
% para gerar as tabelas em LaTeX.
\begin{table}[!htp]
\caption[Legenda curta da tabela]{Legenda longa e mais detalhada da tabela.}
\label{tab:tabela_1}
\begin{center}
\begin{tabular}{cc}
\toprule % Linha superior
Coluna 1 & Coluna2 \\ \midrule % Linha do meio 
a & b \\
c & d \\
e & f \\\bottomrule % Linha inferior
\end{tabular}
\end{center}
\end{table}

As tabelas mais complexas podem ser feitas com ajuda no site \href{http://www.tablesgenerator.com}{http://www.tables\-ge\-ne\-ra\-tor.com}.

Veja um exemplo de equação no próprio texto, \eg, $x=\frac{\sqrt{a^{2}+b^{2}}}{\sigma}$.  Se você preferir, pode escrever a equação de forma estendida, conforme a \Equacao{eq:teste} a seguir. Note que o rótulo é colocado automaticamente.

\begin{equation}
x=\frac{\sqrt{a^{2}+b^{2}}}{\sigma}
\label{eq:teste}
\end{equation}


Você também pode obter ajudar para escrever as equações em \LaTeX no site \url{http://www.codecogs.com/latex/eqneditor.php}.

% Note que podemos incluir automaticamente alguns termos na lista de abreviaturas e siglas no pré-texto. Veja exemplos a seguir no código fonte. 

% A sigla \Sigla{abêcê}{ABC} aparecerá automaticamente na lista de abreviaturas e siglas no pré-texto. Do mesmo modo, podemos usar o comando \SiglaHifen{xisipsilonzê}{XYZ} separado por hífen ao invés de aparecer entre parênteses.

Veja na \Secao{sec:exemplo_secao} a seguir como referenciar uma seção. Também será preciso definir um rótulo (\textit{label}) logo após o comando \texttt{\textbackslash section}.

% Aqui começa uma Seção.
% Use o comando \label para definir um rótulo, 
% caso seja necessário referenciar essa seção
% posteriormente.
\section{Exemplo de seção}\label{sec:exemplo_secao} 
Agora observe como se faz uma citação de artigo científico em periódico \cite{Gradvohl2014c}. De acordo com \textcite{Gradvohl2016}, essa é uma citação direta. Se for citar mais de um trabalho, faça da seguinte forma \cite{Caldana2017,Gradvohl2015}. As referências bibliográficas estão no arquivo \texttt{bibliografia.bib}. Outros exemplos de citações também se encontram nesse arquivo.

Veja a seguir o comando para criar uma figura e o resultado, na \Figura{fig:xwing}. Note, no código fonte, que no comando \texttt{caption} podemos estabelecer uma \enquote{legenda curta} para aparecer na Lista de Figuras. A legenda curta é opcional.

\begin{figure}[!htb]
\centering
% As figuras estão na pasta figuras.
% Se seu texto estiver demorando muito para compilar, use figuras no formato PDF ou PNG.
% Observe, no comando includegraphics que você pode estabelecer uma proporção para a figura.
% Nesse caso, a figura tem uma proporção equivalente à metade (0.5) da largura do texto (\textwidth)
\includegraphics[width=0.5\textwidth]{starwars21280.jpg}
\caption[Legenda curta de figura]{Legenda mais extensa de figura.}
\label{fig:xwing}
\end{figure}

Dica importante: sempre que possível, use figuras no formato PDF. De acordo com o Overleaf, isso acelera a compilação do texto, sem perder a qualidade da figura. 

\subsection{Exemplo de subseção}\label{subsec:exemploSubsec}
É importante evitar chegar a esse nível de subseção. Dois níveis são suficientes. Use essa opção em último caso, apenas.


\subsection{Exemplo de adição de siglas}\label{subsec:siglas}
Para adicionar uma sigla ou abreviatura na lista de siglas e abreviaturas, use o comando ``\texttt{\textbackslash{}Sigla\{nome por extenso\}\{abreviatura\}}'' ou ``\texttt{\textbackslash{}SiglaHifen\{nome por} \texttt{ex\-ten\-so\}\{abreviatura\}}'' para adicionar a sigla com hífen. 
Por exemplo, respectivamente, \Sigla{Ácido Desoxirribonucleico}{DNA} ou \SiglaHifen{Ácido Ribonucleico}{RNA}. A lista de siglas é adicionada automaticamente.

\section{Comandos opcionais para facilitar}\label{sec:comandosOpc}
Este modelo também criou alguns comandos adicionais não apenas para facilitar o trabalho de quem escreve, mas também para manter uma formatação mais consistente.

Entre esses comandos estão o \texttt{\textbackslash{}ie} que inclui a abreviatura ``\ie'' no texto (equivalente ao ``isto é''). Usar esse comando vai garantir que a abreviatura não se separe entre linhas e que o espaço entre o `.' e a próxima letra seja fixo. O mesmo vale para os comandos \texttt{\textbackslash{}eg} que inclui a abreviatura ``\eg'' e \texttt{\textbackslash{}pex} que inclui a abreviatura ``\pex''.

Também existem os comandos \texttt{\textbackslash{}Capitulo\{rótulo do capítulo\}}, \texttt{\textbackslash{}Equacao\{ró\-tu\-lo da equação\}}, \texttt{\textbackslash{}Figura\{rótulo da figura\}}, \texttt{\textbackslash{}Secao\{rótulo da seção\}} e \texttt{\textbackslash{}Tabela\{rótulo da tabela\}}. Esses comandos inserem referências para os respectivos elementos. Além disso, no próprio texto aparece a \textit{string} (``Capítulo'', ``Equação'', ``Figura'' etc) seguida da referência já com o link. Por exemplo, \Secao{sec:exemplo_secao}. Sugere-se a utilização desses comandos para referenciar os respectivos elementos ao invés do comando \texttt{\textbackslash{}ref\{rótulo\}}. Assim, o texto ficará mais uniforme. 

É possível também usar esses comandos nas versões no plural para conjuntos de referências. Por exemplo, para referenciar várias seções, você pode utilizar o comando \texttt{\textbackslash{}secoes\{ró\-tu\-lo\_1, rótulo\_2, rótulo\_3\}}.

Por exemplo, suponha que queiramos referenciar as \secoes{sec:exemplostabelas,sec:exemplo_secao,subsec:siglas}.

% O comando a seguir inclui o arquivo levantamento.tex
% que contém o capítulo de levantamento bibliográfico. 
% Detalhe: não precisa incluir a extensão .tex
\include{levantamento}

% O comando a seguir inclui o arquivo desenvolvimento.tex
% que contém o capítulo de desenvolvimento. 
% Detalhe: não precisa incluir a extensão .tex
\include{desenvolvimento}

% O comando a seguir inclui o arquivo conclusoes.tex
% que contém o capítulo de conclusoes. 
% Detalhe: não precisa incluir a extensão .tex
\include{conclusoes}

% O comando condicional \ifturnitin a seguir é importante para 
% preparar o texto para encaminhamento ao Turnitin. 
% NÃO REMOVA!!
% O \fi correspondente está antes do \end{document}
\ifturnitin
    \relax
\else 
% Comandos para incluir as referências bibliográficas
% Define espaçamento simples em cada referência
\begin{singlespacing}

% Adiciona uma linha em branco entre duas referências
\setlength\bibitemsep{10pt}   
%
% Adiciona as referências bibliográficas.
% Mude o título (title), caso o texto seja em inglês 
% ou espanhol.
\printbibliography[heading=bibintoc, % Adiciona no sumário
                   title={Referências bibliográficas} % Nome do Capítulo
                  ]
\end{singlespacing}

% Os anexos, se houver, vêm depois das referências:
\appendix

% O comando a seguir inclui o arquivo apendices.tex
% que contém os apêndices. Observe o comando \appendix
% na linha anterior
% Detalhe: não precisa incluir a extensão .tex
\include{apendices}
%
%% O comando \fi a seguir é obrigatório para o controle 
%% da opção "turnitin". Não o remova!!
\fi
\end{document}