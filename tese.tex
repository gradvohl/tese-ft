%%%%%%%%%%%%%%%%%%%%%%%%%%%%%%%%%%%%%%%%%%%%%%%%%%%%%%%%%%%%%%%%%%
% The following comments were written in Portuguese, because this 
% template applies only for School of Technology at University 
% of Campinas, Brazil.
%
% Este é um modelo Latex para monografias de Trabalhos de Conclusão 
% de Curso (TCC) na graduação, monografias de Mestrado e Teses de 
% doutorado da Faculdade de Tecnologia (FT) da Universidade 
% Estadual de Campinas (UNICAMP).
%
% Esse modelo e seu respectivo arquivo de classe de documento 
% foram adaptado do modelo de teses e dissertações do 
% Instituto de Computação da UNICAMP.
%
% Autor: André Leon Sampaio Gradvohl, Dr.
% Email:        gradvohl@ft.unicamp.br
% Lattes CV:    http://lattes.cnpq.br/9343261628675642
% 
% Última versão: 28/outubro/2018
%
% Adições/Alterações nesta última versão
% - Pequenos ajustes para se adequar à norma CCPG Nº 002/2018
%%%%%%%%%%%%%%%%%%%%%%%%%%%%%%%%%%%%%%%%%%%%%%%%%%%%%%%%%%%%%%%%%%
%
% Escolha: Portugues ou Ingles ou Espanhol.
% Para a versão final do texto, acrescente a palavra "Final",
% como na segunda linha abaixo da próxima.
%\documentclass[Portugues]{tese-FT}
\documentclass[Portugues,Final]{tese-FT}
%\documentclass[Ingles]{tese-FT}
%\documentclass[Espanhol,Final]{tese-FT}

%Adicione seu arquivo com as referências bibliográficas
\addbibresource{bibliografia.bib}

%O pacote a seguir gera um dummy text. Elimine a linha quando
% for editar seu texto.
\usepackage{lipsum}

\begin{document}

% Escolha entre autor ou autora:
\autor{Nome do autor}
%\autora{Nome da Autora}

% Sempre deve haver um título em português:
\titulo{Título da Dissertação ou Tese em Português}

% Se a língua for o inglês ou o espanhol defina:
%\title{The Dissertation or Thesis Title in English or Spanish for FT}

% Escolha entre orientador ou orientadora e inclua os títulos:
\orientador{Prof. Dr. Nome do Orientador}
%\orientadora{Profa. Dra. Nome da Orientadora}

% Escolha entre coorientador ou coorientadora, se houver, 
% e inclua os títulos:
%\coorientador{Prof. Dr. Eng. Lic. Nome do Co-Orientador}
%\coorientadora{Prof. Dra. Eng. Lic. Nome da Co-Orientadora}

% Escolha entre uma das quatro opções a seguir (comente as demais):
%\bsi         % para Trabalho de Conclusão de Curso em BSI
%\tads       % para Trabalho de Conclusão de Curso em TADS
%\qualificacaoMestrado  % Para textos de qualificação de mestrado.
%\qualificacaoDoutorado % Para textos de qualificação de doutorado.
%\mestrado   % para Dissertação de Mestrado em Tecnologia
\doutorado  % para Tese de Doutorado em Tecnologia

%Defina a área de concentração. Se for TCC, deixe comentado
\areaConcentracao{Sistemas de Informação e Comunicação}
%\areaConcentracao{Ambiente}
%\areaConcentracao{Ciência dos Materiais}

% Se houve cotutela, defina:
%\cotutela{Universidade Nova de Plutão}

%Defina a data da defesa no formato {Dia}{Mês}{Ano}
\datadadefesa{16}{07}{2018}

% Para a versão final defina:
% Repita o nome do Orientador(a) no primeiro avaliador
\avaliadorA{Prof. Dr. Nome do Orientador}{FT/UNICAMP}
\avaliadorB{Profa. Dra. Segunda Avaliadora}{Instituição da segunda avaliadora}
\avaliadorC{Dr. Terceiro Avaliador}{Instituição do terceiro avaliador}
%\avaliadorD{Prof. Dr. Quarto Avaliador}{Instituição do quarto avaliador}
%\avaliadorE{Prof. Dr. Quinto Avaliador}{Instituição do quinto avaliador}
%\avaliadorF{Prof. Dr. Sexto Avaliador}{Instituição do sexto avaliador}
%\avaliadorG{Prof. Dr. Sétimo Avaliador}{Instituição do sétimo avaliador}
%\avaliadorH{Prof. Dr. Oitavo Avaliador}{Instituição do oitavo avaliador}


% Para incluir a ficha catalográfica em PDF na versão final, 
% copie o arquivo PDF, descomente e ajuste a linha a seguir:
%\fichacatalografica{arquivo.pdf}

% Este comando deve ficar aqui:
\paginasiniciais

% Se houver dedicatória, descomente a linha a seguir:
%\prefacesection{Isso é um teste}
%A dedicatória deve ocupar uma única página.
%
%
% Se houver epígrafe, descomente e edite as linhas a seguir:
% \begin{epigrafe}
% {\it
% Vita brevis,\\
% ars longa,\\
% occasio praeceps,\\
% experimentum periculosum,\\
% iudicium difficile.}
%
% \hfill (Hippocrates)
% \end{epigrafe}
%
%
% Adicione no arquivo "agradecimentos.tex" os seus agradecimentos
% Caso prefira omitir os agradecimentos, comente a linha a seguir.
\prefacesection{Agradecimentos}
Coloque nesse arquivo os agradecimentos àqueles que o ajudaram no seu trabalho. Os agradecimentos devem ocupar uma única página. Não esqueça de adicionar a frase a seguir.

O presente trabalho foi realizado com apoio da Coordenação de Aperfeiçoamento de Pessoal de Nível Superior -- Brasil (CAPES) -- Código de Financiamento 001.

% Sempre deve haver um resumo em português:
\begin{resumo}
O resumo deve ter no máximo 500 palavras e deve ocupar uma única página.
\end{resumo}

% Sempre deve haver um abstract:
\begin{abstract}
The abstract must have at most 500 words and must fit in a single page.
\end{abstract}

% Se houver um resumo em espanhol, descomente as linhas a seguir:
%\begin{resumen}
% A mesma regra aplica-se.
%\end{resumen}

% A lista de figuras:
\listoffigures

% A lista de tabelas:
\listoftables

% A lista de abreviações e siglas é opcional:
% \prefacesection{Lista de Abreviações e Siglas}

% A lista de símbolos é opcional:
% \prefacesection{Lista de Símbolos}

% A lista de abreviações e siglas vem a seguir.
% Dê uma olhada no pacote nomencl para ver os comandos para 
% adicionar abreviações e siglas no texto.
\renewcommand{\nomname}{Lista de Abreviações e Siglas}
\printnomenclature[3cm]

% O sumário vem aqui:
\tableofcontents

% E esta linha deve ficar bem aqui:
\fimdaspaginasiniciais

% O corpo da dissertação ou tese começa aqui:
%
% O comando a seguir inclui o arquivo introducao.tex
% que contém o capítulo de Introdução. 
% Detalhe: não precisa incluir a extensão .tex
% Aqui começa o capítulo de Introdução.
% Use o comando \label para definir um rótulo, 
% caso seja necessário referenciar esse capítulo
% posteriormente.
\chapter{Introdução}\label{chp:Introducao}

% O comando a seguir gera um "dummy text". 
% Elimine-o quando escrever sua dissertação.
\lipsum[1]

Aqui um exemplo de como referenciar a \Tabela{tab:tabela_1}. Note que é preciso definir um rótulo (\textit{label}) dentro do comando de definição da tabela.

% Veja a seguir um exemplo de Tabela.
% Você pode usar o site http://www.tablesgenerator.com
% para gerar as tabelas em LaTeX.
\begin{table}[!htp]
\caption[Legenda curta da tabela]{Legenda longa e mais detalhada da tabela.}
\label{tab:tabela_1}
\begin{center}
\begin{tabular}{|c|c|}
\hline
Coluna 1 & Coluna2 \\ \hline\hline
a & b \\\hline
c & d \\\hline
\end{tabular}
\end{center}
\end{table}

As tabelas mais complexas podem ser feitas com a ajuda no site \url{http://www.tablesgenerator.com}.

Veja um exemplo de equação no próprio texto, e.g., $x=\frac{\sqrt[y]{a^{2}+b^{2}}}{\sigma}$.  Se você preferir, pode escrever a equação de forma estendida, conforme a \Equacao{eq:teste} a seguir. Note que o rótulo é colocado automaticamente.

\begin{equation}
x=\frac{\sqrt[y]{a^{2}+b^{2}}}{\sigma}
\label{eq:teste}
\end{equation}

Você também pode obter ajudar para escrever as equações em \LaTeX no site \url{http://www.codecogs.com/latex/eqneditor.php}.

Note que podemos incluir automaticamente alguns termos na lista de símbolos no pré-texto. Veja exemplos a seguir no código fonte. Eles não aparecerão no arquivo gerado. 

\nomenclature{$a$}{Valor obtido a partir de um termo}%
\nomenclature{$b$}{Outro valor obtido a partir de outro termo}%

Veja na \Secao{sec:exemplo_secao} a seguir como referenciar uma seção. Também será preciso definir um rótulo (\textit{label}) logo após o comando \texttt{$\backslash$section}.

% Aqui começa uma Seção.
% Use o comando \label para definir um rótulo, 
% caso seja necessário referenciar essa seção
% posteriormente.
\section{Exemplo de seção}\label{sec:exemplo_secao} 
Agora observe como se faz uma citação de artigo científico em periódico \cite{Gradvohl2014c}. De acordo com \citeonline{Gradvohl2016}, essa é uma citação direta. Se for citar mais de um autor, faça da seguinte forma \cite{ZAMBON2015, Gradvohl2015}. As referências bibliográficas estão no arquivo \texttt{bibliografia.bib}

Veja a seguir o comando para criar uma figura e o resultado, na \Figura{fig:xwing}.

\begin{figure}[!htb]
\centering
%As figuras estão na pasta figuras.
\includegraphics[scale=0.2]{starwars21280.jpg}
\caption[Legenda curta de figura]{Legenda mais extensa de figura.}
\label{fig:xwing}
\end{figure}

\subsection{Exemplo de subseção}
% O comando a seguir gera um "dummy text". 
% Elimine-o quando escrever sua dissertação.
\lipsum[5]


% O comando a seguir inclui o arquivo levantamento.tex
% que contém o capítulo de levantamento bibliográfico. 
% Detalhe: não precisa incluir a extensão .tex
\chapter{Levantamento bibliográfico}\label{chp:levantamento}
% O comando a seguir gera um "dummy text". 
% Elimine-o quando escrever sua dissertação.
\lipsum[6]


% O comando a seguir inclui o arquivo desenvolvimento.tex
% que contém o capítulo de desenvolvimento. 
% Detalhe: não precisa incluir a extensão .tex
\chapter{Desenvolvimento}\label{chp:desenvolvimento}
% O comando a seguir gera um "dummy text". 
% Elimine-o quando escrever sua dissertação.
\lipsum[7]


% O comando a seguir inclui o arquivo conclusoes.tex
% que contém o capítulo de conclusoes. 
% Detalhe: não precisa incluir a extensão .tex
\chapter{Conclusões}\label{chp:conclusoes}
% O comando a seguir gera um "dummy text". 
% Elimine-o quando escrever sua dissertação.
\lipsum[8]

% Os comandos para incluir as referências bibliográficas
\printbibliography[heading=bibintoc, % Adiciona no sumário
                   title=Referências bibliográficas % Nome do Capítulo
                  ]

% Os anexos, se houver, vêm depois das referências:
\appendix

% O comando a seguir inclui o arquivo apendices.tex
% que contém os apêndices. Observe o comando \appendix
% na linha anterior
% Detalhe: não precisa incluir a extensão .tex
\chapter{Primeiro Apêndice}
% O comando a seguir gera um "dummy text". 
% Elimine-o quando escrever sua dissertação.
\lipsum[9]

\chapter{Segundo  Apêndice}
% O comando a seguir gera um "dummy text". 
% Elimine-o quando escrever sua dissertação.
\lipsum[10]


\end{document}